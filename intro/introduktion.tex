\chapter{Velkommen}

Velkommen til denne lille LaTeX introduktion!  LaTeX er et sprog til at skrive rapporter, bøger, manualer og andre tekster i. Resten af dette kapitel handler om at sætte LaTeX op. Læs hvad du har lyst til i dette kapitel, og \textbf{åben så "latexintroduktion.tex"} dokumentet i en LaTeX editor eller bare i et tekst-program. Når du har læst det færdig, kan du fortsætte i denne pdf. Dele af dette dokument er kun brugbart for nogle - hvis du ikke skal skrive formler, så spring det kapitel over.\\

\noindent
Nu skal du åbne "latexintroduktion.tex"!

\section{Hvorfor LaTeX}
Hvorfor bruge LaTeX? Jeg vil ikke skrive en salgstale her. LaTeX er god på grund af tre ting: Det adskiller design og tekst, det holder styr på alting for dig, og hvis det ikke kan det du skal bruge, er der altid en eller anden som har haft brug for det samme, og har skrevet en udviddelse til LaTeX med det. Brug LaTeX til store rapporter/artikler og manualer. Bøger, mindre artikler og anden udgiven tekst virker også fint. Jeg vil ikke anbefale LaTeX til noter.

Hvorfor ikke LaTeX? Hvis du ikke kan lide LaTeX's layout eller design skal du skrive kode for at lave det om. Du kan enten søge på nettet om der er gode designs derude (det er der), eller du kan lære LaTeX's layout-sprog selv, men det handler denne introduktion ikke om.

\section{TeXstudio}
For at kunne arbejde ordentligt i LaTeX, er det en god idé at bruge en LaTeX-editor (man kan godt lade være, men det vil jeg ikke hjælpe dig med :) ). Jeg har selv mest arbejdet med TeXstudio, og det findes til windows, mac og linux, så det er det jeg foreslår. Så kan du altid skifte senere, når du forstår LaTeX bedre :)

TeXstudio finder du på \url{http://texstudio.sourceforge.net/}. Installer det. For at følge denne guide foreslår jeg at du gør følgende ting:
\begin{itemize}
\item Sæt TeXstudio op til at bruge biblatex.
\begin{itemize}
\item Gå til settings (windows: ctrl + ,) (mac: cmd + ,).
\item Gå til "Build".
\item Under "Default Bibliography" vælg biblatex eller biber.
\end{itemize}
\item Brug UTF-8 encoding. Ellers kan du ikke bruge æ, ø og å.
\begin{itemize}
\item Gå til settings.
\item Gå til "Editor".
\item Under "Default Text Encoding" vælg UTF-8.
\end{itemize}
\end{itemize}

Hvis du vil checke om dit nuværende dokument bruger UTF-8, så kig i nederste højre hjørne. Andre muligheder er ASCII, ISO-5589-1 eller Latin-1, men brug altid UTF-8. Her kan du også ændre stavekontrollens sprog.

\section{Andre småting}
Denne introduktion præsenterer kun de mest brugte funktioner i LaTeX, og tager ikke meget hensyn til fejl. For en langt mere detaljeret introduktion, læs \url{http://en.wikibooks.org/wiki/LaTeX}.

\subsubsection{Fejl}
Hvis der opstår fejl med noget jeg beder dig om, så se i bilagene om jeg har beskrevet den, ellers søg på nettet. De fleste fejl er lette at rette.

\subsubsection{Tak til}
Tak til - liste og \todo{takliste}
