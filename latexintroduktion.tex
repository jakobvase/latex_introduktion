%Velkommen til LaTeX introduktionen!
%Denne linie er en kommentar, det kan du se ved at jeg bruger "%" tegnet i starten af linien. Den bliver ikke vist i det endelige dokument.
%Jeg bruger kommentarer til at forklare de ting der står i dette dokument.


%Det første vi skal er at sige at dette er et LaTeX dokument. Det skriver vi således:
\documentclass{report}

%Herefter sætter vi LaTeX til at være på dansk, og til at bruge utf8 format - det gør vi for at kunne bruge Æ, Ø og Å.
\usepackage[danish]{babel}
\usepackage[utf8]{inputenc}


%Alle kommandoer i LaTeX starter med \-tegnet. Det kan du se i de tre øverste kommandoer her.
%LaTeX er sejt, fordi man kan tilføje nye funktioner til det. Det har vi alledere gjort 2 gange, med kommandoen \usepackage{}. Inde imellem {} skriver man den pakke man skal bruge.
%Hvis du, efter at have læst denne introduktion, har brug for noget som jeg ikke viser her, så bare søg på nettet efter det, der findes virkelig mange pakker med alt muligt underligt :)
%Nu tilføjer jeg lige 5 pakker mere, som jeg selv altid har med, og som jeg vil forklare hvordan man bruger undervejs:


\usepackage{hyperref}
%Pakken hyperref gør alle referencer i rapporten til links man kan klikke på. Så indholdsfortegnelsen kan altså bruges til at hoppe ned til det sted man gerne vil hen.

\usepackage{graphicx}
%Pakken graphicx er rigtig god til at indsætte billeder med.

\usepackage{pdfpages}
%Pakken pdfpages er rigtig god til at indsætte andre pdf-filer med. Det har man ofte brug for i bilag.

\usepackage{supertabular}
%Pakken supertabular er god til at lave tabeller - LaTeX kan normalt ikke klare tabeller som går over 1 side, men det kan supertabular.

\usepackage{float}
%Pakken float er godt til at styre hvor billeder kommer hen - det får du at se senere.

\usepackage{biblatex}
\addbibresource{examplebib.bib}
%Pakken biblatex er til at lave litteraturlister. Den er smart, og det viser jeg senere hvorfor. \addbibresource bruges til at tilføje en litteraturliste-fil, det beskriver jeg i samme afsnit.

\usepackage{amsmath}
%Pakken amsmath har en masse dejlige funktioner til at skrive flottere matematiske formler.

\usepackage{todonotes}

%Mange pakker kan man give ekstra argumenter til. Det gør man ved at skrive: \usepackage[argument]{package}. Mange af disse pakker har masser af indstillingsmuligheder, men jeg bruger det sjældent. Det eneste sted jeg bruger argumenter er i \documentclass{report}, i toppen af dokumentet. Her skriver jeg ofte: \documentclass[hidelinks, a4paper]{report}, fordi jeg ikke syns at pdf-links farver er pæne, og for at sikre at papiret er i a4 format. Jeg har ladt dem ude af denne introduktion for eksemplets skyld. Min standardforside ligger også i mappen her, og hedder standardforside.tex.


%For at gøre tabeller lidt pænere tilføjer jeg ofte denne linie:
%\renewcommand{\arraystretch}{1.5}
%Den gør, at tabeller og lister får lidt mere mellemrum mellem linierne, og det kan jeg godt lide. Du kan dekommentere den ved at fjerne % tegnet og se det.

%Endelig kan man tilføje sine egne kommandoer til LaTeX, som denne her:
\newcommand{\bs}{\textbackslash}
%\newcommand er kommandoen til at tilføje sin egen kommando. Det første i {}'erne er navnet på kommandoen, (her \bs), det andet er hvad kommandoen gør. Kommandoen \textbackslash printer \-tegnet, og det skal jeg bruge meget i denne introduktion, så jeg hjælper bare mig selv lidt her :)
%Jeg tilføjer ofte også denne kommando:
\newcommand{\myquote}[1]{``{#1}''}
%Her er der et argument med til kommandoen, [1], som jeg så kan bruge med {#1} i udførelsen. \myquote{LaTeX} skriver altså "LaTeX". (`` og '' bliver lavet om så de vender den rigtige vej i dokumentet.)


%Så er vi færdige med setup.
%Nu er der kun tilbage at begynde dokumentet! Here we go :)
\begin{document}

%For at fortælle LaTeX hvad dokumentet hedder bruger man:
\title{En introduktion til \LaTeX}

%For at fortælle LaTeX hvem forfatteren er bruger man:
\author{Jakob Ambeck Vase}
%Hvis der er flere end en forfatter bruger man \and imellem: \author{Jakob \and Magnus \and Matilde}

%Og for at lave en forside:
\maketitle

%De to kommandoer herunder laver hhv. en indholdsfortegnelse og en liste af figurer i dokumentet. Jeg har udkommenteret dem, for de er ikke nødvendige i denne introduktion.
%\tableofcontents
%\listoffigures
\listoftodos


%Nu kommer vi så til den vigtigste kommando i LaTeX, den hvor du importerer andre filer:
\chapter{Velkommen}

Velkommen til denne lille LaTeX introduktion!  LaTeX er et sprog til at skrive rapporter, bøger, manualer og andre tekster i. Resten af dette kapitel handler om at sætte LaTeX op. Læs hvad du har lyst til i dette kapitel, og \textbf{åben så "latexintroduktion.tex"} dokumentet i en LaTeX editor eller bare i et tekst-program. Når du har læst det færdig, kan du fortsætte i denne pdf. Dele af dette dokument er kun brugbart for nogle - hvis du ikke skal skrive formler, så spring det kapitel over.\\

\noindent
Nu skal du åbne "latexintroduktion.tex"!

\section{Hvorfor LaTeX}
Hvorfor bruge LaTeX? Jeg vil ikke skrive en salgstale her. LaTeX er god på grund af tre ting: Det adskiller design og tekst, det holder styr på alting for dig, og hvis det ikke kan det du skal bruge, er der altid en eller anden som har haft brug for det samme, og har skrevet en udviddelse til LaTeX med det. Brug LaTeX til store rapporter/artikler og manualer. Bøger, mindre artikler og anden udgiven tekst virker også fint. Jeg vil ikke anbefale LaTeX til noter.

Hvorfor ikke LaTeX? Hvis du ikke kan lide LaTeX's layout eller design skal du skrive kode for at lave det om. Du kan enten søge på nettet om der er gode designs derude (det er der), eller du kan lære LaTeX's layout-sprog selv, men det handler denne introduktion ikke om.

\section{TeXstudio}
For at kunne arbejde ordentligt i LaTeX, er det en god idé at bruge en LaTeX-editor (man kan godt lade være, men det vil jeg ikke hjælpe dig med :) ). Jeg har selv mest arbejdet med TeXstudio, og det findes til windows, mac og linux, så det er det jeg foreslår. Så kan du altid skifte senere, når du forstår LaTeX bedre :)

TeXstudio finder du på \url{http://texstudio.sourceforge.net/}. Installer det. For at følge denne guide foreslår jeg at du gør følgende ting:
\begin{itemize}
\item Sæt TeXstudio op til at bruge biblatex.
\begin{itemize}
\item Gå til settings (windows: ctrl + ,) (mac: cmd + ,).
\item Gå til "Build".
\item Under "Default Bibliography" vælg biblatex eller biber.
\end{itemize}
\end{itemize}

%Den henter filen introduktion.tex i mappen intro. Dette er smart til at strukturere dine dokumenter, som du kan se længere nede.


%Du har nu fået en introduktion til hvordan man sætter et LaTeX dokument op.
%Her følger resten af kapitlerne i denne introduktion, og du kan gå tilbage til at læse pdf'en. (De er også forklarede, men ikke så grundigt som her :) )

\chapter{Standard LaTeX kommandoer}
Nu hvor du har læst baggrundsfilen, er du klar til at begive dig i kast med at skrive. Det første du skal vide er, hvordan man laver kapitler i LaTeX:\\ 
%\\ laver en ny linie. Jeg vil gerne have eksemplerne til at stå klarere frem.

\bs chapter\{Kapitlets navn\}\\

\noindent %LaTeX indsætter som standard indentering i starten af hver paragraf. Denne kommando fjerner den.
Dette printer det samme som du kan se i toppen af denne side, og hvert kapitel kommer på en ny side.

\noindent
Hvis du gerne vil have sektioner i dine kapitler, bruger du disse:\\

\bs section\{Sektionens navn\}

\section{Sektion}
LaTeX giver dem selv numre, så de passer ind i dokumentet (og hvis du senere rykker rundt, passer numrene stadig.)

\noindent
LaTeX kan også lave sub- og subsubsections, men ikke lavere end dette uden ekstra pakker:\\

\bs subsection\{Undersektion-navn\}

\subsection{Undersektion}
Og endelig:\\

\bs subsubsection\{Under-undersektion-navn\}

\subsubsection{Under-undersektion}
Den her har slet ikke noget tal :)

\section{Kursiv, fed, understreget}
LaTeX har en standardkommando, for hvis du gerne vil forstærke tekst, som hedder \bs emph\{text\}. Det ser \emph{således} ud. Hvis du gerne vil bruge \emph{flere} metoder, så kan du bruge \bs textbf\{tekst\} for \textbf{fed tekst}, \bs textit\{tekst\} for \textit{kursiv tekst} og \bs underline\{tekst\} for \underline{understreget tekst}. Ellers ville jeg anbefale bare at bruge \bs emph, for den kan ændres, hvis du på et tidspunkt gerne vil vise dine markeringer på en anden måde.

\section{Lister}
I LaTeX laves lister således:\\

\indent \bs begin\{itemize\}\\
\indent \bs item Første punkt på listen.\\
\indent \bs item Andet punkt på listen.\\
\indent \bs end\{itemize\}\\

\noindent
Det er vigtigt både at skrive \emph{begin} og \emph{end}. Denne ser således ud:

\begin{itemize}
\item Første punkt på listen.
\item Andet punkt på listen.
\end{itemize}

\noindent
Man kan også lave nummererede lister:\\

\indent \bs begin\{enumerate\}\\
\indent \bs item Første punkt på listen\\
\indent \bs item Andet punkt på listen\\
\indent \bs end\{enumerate\}\\

\noindent
Det ser således ud:

\begin{enumerate}
\item Første punkt på listen
\item Andet punkt på listen
\end{enumerate}

\noindent
Man kan også lave flere lister inde i hinanden, altså give punkter underpunkter:\\

\indent \bs begin\{itemize\}\\
\indent \bs item Første punkt på listen\\
\indent \bs item Andet punkt på listen\\
\indent \bs begin\{itemize\}\\
\indent \bs item Første underpunkt på andet punkt\\
\indent \bs item Andet underpunkt på andet punkt\\
\indent \bs end\{itemize\}\\
\indent \bs item Tredje punkt på listen\\
\indent \bs end\{itemize\}\\

\begin{itemize}
\item Første punkt på listen.
\item Andet punkt på listen.
\begin{itemize}
\item Første underpunkt på andet punkt.
\item Andet underpunkt på andet punkt.
\end{itemize}
\item Tredje punkt på listen.
\end{itemize}

\noindent
I denne situation er det vigtigt at holde styr på, hvornår man skriver \bs begin og \bs end. LaTeX fortæller dig hvis du laver fejl, men ikke hvis underpunkterne er det forkerte sted :)\\

\noindent
Endelig kan man skrive tekst i stedet for prikkerne således:\\

\indent \bs begin\{itemize\}\\
\indent \bs item[Se] Det man laver med øjnene.\\
\indent \bs item[Høre] Det man laver med ørerne.\\
\indent \bs end\{itemize\}\\

\noindent
Det ser således ud:

\begin{itemize}
\item[Se] Det man laver med øjnene.
\item[Høre] Det man laver med ørerne.
\end{itemize}

\noindent
Det var alt hvad jeg havde om lister :) %De ting man ofte bruger mens man skriver.
\chapter{Referencer}
Referencer i LaTeX virker på en anden måde end du måske er vant til. Det kan gøres på forskellige måder. Den første handler om når du refererer dig selv andre steder i dokumentet.
\label{chap:referencer}

\section{Selvreferencer}
Når man gerne vil referere sig selv (som for eksempel i sætningen \myquote{Se f.eks. indledningen (side 3)}), bruger man i LaTeX to kommandoer. En til at markere et sted som noget der skal refereres til:\\

\bs label\{introduktion\}\\

\noindent
Og et til at referere til det:\\

\bs ref\{introduktion\}\\

\noindent
Ordet \myquote{introduktion}, som jeg skriver inden i \bs label og \bs ref, er det navn, som jeg husker stedet med. Så hvis jeg gerne vil referere til introduktionen, giver det mening at kalde den det. Hvis jeg ville referere til problemformuleringen i stedet, ville jeg kalde den det.

\noindent
Nu vil jeg give et eksempel:\\

\indent Her skriver jeg \bs label\{selvreferencer\}: \label{selvreferencer}\\
\indent Her skriver jeg \bs ref\{selvreferencer\}: \ref{selvreferencer}\\

\noindent
Ved \bs label står der ingenting, og ved \bs ref står sektionsnummeret der. Så nu kan jeg altså referere til denne sektion et hvilket som helst sted i dokumentet, ved at skrive således:\\

\indent Jeg skriver om referencer i sektion \bs ref\{selvreferencer\}\\
\indent Jeg skriver om referencer i sektion \ref{selvreferencer}\\

\noindent
Man kan også referere til sidetallet ved at skrive \bs pageref:\\

\indent Jeg skriver om selvreferencer på side \bs pageref\{selvreferencer\}\\
\indent Jeg skriver om selvreferencer på side \pageref{selvreferencer}\\

\noindent
Smart! Det eneste man skal være opmærksom på, er at give \bs label'sene forskellige navne, for ellers ved LaTeX ikke hvilken én man mener.

\section{Fodnoter, kommentarer og slutnoter}
Fodnoter indsættes ved at skrive:\\

\bs footnote\{Tekst som skal stå i fodnoten.\}\\

\noindent
Kommentarer indsættes ved at skrive:\\

\bs marginpar\{Tekst som skal stå i kommentaren.\}\\

\noindent
Her kommer eksempler på de to:\\

\indent Her skriver jeg en fodnote.\bs footnote\{Dette er en fodnote.\}\\
\indent Her skriver jeg en fodnote.\footnote{Dette er en fodnote.}\\
\indent Her skriver jeg en kommentar.\bs marginpar\{Dette er en kommentar.\}\\
\indent Her skriver jeg en kommentar.\marginpar{Dette er en kommentar.}\\

\noindent
Slutnoter er lidt mere omstændige, der skal man hente pakken \myquote{endnotes}, og derefter skrive denne kommando:\\

\bs let\bs footnote=\bs endnote\\

\noindent
Så altså, for at bruge slutnoter:\\

\indent \bs usepackage\{endnotes\}\\
\indent \bs let\bs footnote=\bs endnote\\

\noindent
Skal sættes ind i hovedfilen inden dokumentet begynder. %Hvordan man refererer til andre steder i dokumentet, laver fodnoter, slutnoter og litteraturreferencer, og hvordan man laver links.
\section{Litteraturliste}

LaTeX er lavet til folk, som skriver mange artikler og bøger. De har brug for et smart system til at styre litteraturlister. I LaTeX skrive man bøger, artikler og hjemmesider ind i en fil for sig, og refererer så til dem ved hjælp af nøgleord. LaTeX holder så selv styr på hvilke emner du har refereret til, og laver en litteraturliste der kun indeholder dem. Tanken er, at du kan bruge den samme litteraturliste-fil til mange forskellige dokumenter.\\

\subsection{Litteraturliste-filen}
Litteraturliste-filer slutter med ".bib". Du kan åbne examplebib.bib i mappen, og se hvordan de er skrevet. Det vigtige er, at hvert emne har et nøgleord som jeg bruger her.

\subsection{Citation}
I LaTeX ser en citation således ud:\\

\bs autocite\{nøgleord\}\\

\noindent
Dette printer følgende:\\

\autocite{vase14}\\

\noindent




\nocite{john3}
[autocite=plain,inline,footnote,superscript] %Hvad autocite skal gøre.
[backref] %Om der skal printes en liste af steder hvor emnet i litteraturlisten bliver brugt.
[citationstyle=authoryear, numeric] Hvis man gerne vil ændre hvordan citater ser ud.
[bibliographystyle=authoryear, numeric] Hvis man gerne vil ændre hvordan litteraturlisten ser ud.
 %Dette er fordi litteraturliste forklaringen er ret lang.
\chapter{Formler}
\todo{Formler og math mode} %Hvordan man skriver formler i LaTeX.

\chapter{Billeder}

At indsætte billeder i LaTeX er rimelig simpelt. Jeg vil her skrive den mest enkle måde, og derefter give et par muligheder for ændringer:\\

\indent \bs begin\{figure\}\\
\indent \bs includegraphics[width=1\bs linewidth]\{sti til billedet\}\\
\indent \bs end\{figure\}\\

\noindent
\bs begin og \bs end fortæller LaTeX at her skal vises en figur. Det er altid en god ide at have dem med, fordi billedet så bliver behandlet som en figur. \bs includegraphics er den kommando som faktisk henter billedet.\footnote{\bs includegraphics kommer fra pakken \emph{graphicx}.} [width=1\bs linewidth] fortæller LaTeX hvor bredt billedet må være (det betyder "1 * linjebredden"), her kan du selv eksperimentere med hvad der ser godt ud. \bs linewidth er hvor bred teksten er i dit dokument.\\
\noindent Her får du et eksempel:\\

\indent \bs begin\{figure\}\\
\indent \bs includegraphics[width=1\bs linewidth]\{billeder/Madonna.jpg\}\\
\indent \bs end\{figure\}\\

\begin{figure}[H]
\includegraphics[width=1\linewidth]{billeder/Madonna.jpg}
\end{figure}

\section{Standard stuff}

Når jeg indsætter billeder, gør jeg som standard således:\\

\indent \bs begin\{figure\}[H]\\
\indent \bs centering\\
\indent \bs includegraphics[width=1\bs linewidth]\{sti\}\\
\indent \bs caption\{billedtekst\}\\
\indent \bs label\{fig:billedlabel\}\\
\indent \bs end\{figure\}\\

\noindent
[H] siger til LaTeX at billedet skal være der hvor jeg skriver det, og ikke må rykkes rundt.\footnote{[H] kommer fra pakken \emph{float}.} Normalt sætter LaTeX billeder således at siderne kan fyldes helt ud med tekst.

\noindent
\bs centering gør at LaTeX sætter billedet på midten af siden - du kan ikke se det i disse eksempler, men hvis billedet havde været mindre havde det gjort en forskel.

\noindent
\bs caption\{billedtekst\} sætter en tekst under billedet. Den gør også at billedet får vist sit tal.

\noindent
\bs label\{fig:billedlabel\} gør det nemmere at referere til billedet - se sektion  \ref{selvreferencer}. Når man giver figurer referencer, er det en god ide at skrive fig: foran - så er det nemmere at se hvad det er man refererer.\\

\noindent
Alt i alt ser det således ud:

\begin{figure}[H]
\centering
\includegraphics[width=1\linewidth]{billeder/Madonna.jpg}
\caption{"Madonna of the Rocks" af Leonardo da Vinci}
\label{fig:madonna_standard}
\end{figure}

\noindent
Den eneste forskel du kan se er altså figurteksten nedenunder :)

\section{Andet billedlayout}
\subsubsection{Rammer}
Hvis du gerne vil have en ramme rundt om et billede, kan du skrive \bs includegraphics ind i en \bs fbox:\\

\indent \bs begin\{figure\}\\
\indent ... \\
\indent \bs fbox\{\bs includegraphics[width=.8\bs linewidth]\{sti til billede\}\}\\
\indent ... \\
\indent \bs end\{figure\}\\

\begin{figure}[H]
\centering
\fbox{\includegraphics[width=.8\linewidth]{billeder/Madonna.jpg}}
\caption{Madonna med ramme}
\label{fig:madonna_ramme}
\end{figure}

\noindent
Hvis du vil have mere avancerede rammer, vil jeg foreslå at du søger på nettet. Der findes en pakke som hedder \emph{mdframed} som er sej, men det er for meget at skrive om her.

\subsubsection{Text-wrap}
At få teksten til at kunne stå rundt om billedet er en lidt større operation, og sjældent nødvendig i akademiske tekster. Se dette link for en beskrivelse af hvordan man kan gøre: \url{http://en.wikibooks.org/wiki/LaTeX/Floats,_Figures_and_Captions}.

\section{PDF'er}
Hvis man gerne vil inkludere en pdf fil eller et par sider fra den, skal man bruge følgende kommando:\\

\indent \bs includepdf[pages=\{sidetal\}]\{sti til pdf'en\}\\

\noindent
I "sidetal" feltet kan du skrive, for eksempel \{1, 3, 5-9\}, og du vil i så fald få side 1, 3 og 5 til 9 med. Hvis du vil have alle siderne med, skriver du \{-\}. Pdf sider vil altid stå på en side for sig.\footnote{\bs includepdf kommandoen kommer fra pakken \emph{pdfpages}.}
 %Hvordan man indsætter billeder og pdf'er.
\chapter{Tabeller}

\todo{Skriv om hvordan med tabeller.} %Hvordan man laver tabeller.
\todo{Kig på Latex environments}

%Herefter vil de fleste gerne skrive deres litteraturliste:
\printbibliography

%Når man så har skrevet sine kapitler, kan man bruge
\appendix
%kommandoen, til at fortælle LaTeX at nu begynder bilagene. Alle \chapter kommandoer herefter bliver lavet om til bilag.
\chapter{Bilagseksempel}

Total standard bilag.\todo{Skriv noget ordentligt i bilaget.}

%Denne kommando viser LaTeX, at dokumentet er forbi.
\end{document}
%Tak for at du har fulgt denne introduktion. Nu kan du klare basis, og kan søge dig frem til resten på nettet.