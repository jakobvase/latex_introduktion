\chapter{Formler}
\label{chap:formler}
LaTeX blev oprindeligt lavet, fordi matematikeren Donald Knuth manglede et program til at skrive flotte formler med. Så LaTeX er ret sej til formler.

\section{Math mode}
I LaTeX skelner man mellem "text mode" og "math mode". Normalt når du skriver er du i "text mode", og når du gerne vil skrive formler skal du i "math mode". Der findes to slags math mode, "inline" og "display". Jeg viser dem her:\\

\indent Denne formel er inline $2 + 3 = 5$, og nu skriver jeg videre.\\
\indent Denne formel er display \[2 + 3 = 5\]\indent, og nu skriver jeg videre.\\

\noindent For at gå i inline math mode skriver du \$ formel \$, og for at gå i display math mode skriver du \bs [ formel \bs ].

For at skrive tekst inde i math mode (som for eksempel i integralregning), bruger man kommandoen \bs text\{den tekst du vil skrive\}. (Du kan også bruge \bs textbf, \bs textit osv.)\\

\noindent
I resten af kapitlet vil jeg gå ud fra at du er i math mode.

\section{Matematiske symboler}

\todo{Matematiske symboler, opløftet, osv.}

For en dyberegående diskussion af formler i LaTeX, check \url{http://en.wikibooks.org/wiki/LaTeX/Mathematics}, og for endnu dybere diskussion \url{http://en.wikibooks.org/wiki/LaTeX/Advanced_Mathematics}.