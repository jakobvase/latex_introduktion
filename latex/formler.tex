\chapter{Formler}
\label{chap:formler}
LaTeX blev oprindeligt lavet, fordi matematikeren Donald Knuth manglede et program til at skrive flotte formler med. Så LaTeX er ret sej til formler.

\section{Math mode}
I LaTeX skelner man mellem "text mode" og "math mode". Normalt når du skriver er du i "text mode", og når du gerne vil skrive formler skal du i "math mode". Der findes to slags math mode, "inline" og "display". Jeg viser dem her:\\

\indent Denne formel er inline: $2 + 3 = 5$, og nu skriver jeg videre.\\
\indent Denne formel er display: \[2 + 3 = 5\]\indent, og nu skriver jeg videre.\\

\noindent For at gå i inline math mode skriver du \$ formel \$, og for at gå i display math mode skriver du \bs [ formel \bs ].

For at skrive almindelig, ikke-kursiv tekst inde i math mode (som for eksempel i integralregning), bruger man kommandoen \bs textrm\{den tekst du vil skrive\}. (Du kan også bruge \bs textbf, \bs textit osv.)\\

\noindent
I resten af kapitlet vil jeg gå ud fra at du er i math mode.

\section{Matematisk notation}
Her følger en liste af de mest brugte matematiske notationer. For en dyberegående diskussion af formler i LaTeX, check \url{http://en.wikibooks.org/wiki/LaTeX/Mathematics}, og for endnu dybere diskussion \url{http://en.wikibooks.org/wiki/LaTeX/Advanced_Mathematics}.\\

\noindent
Alle de følgende notationer kan blandes og mixes, som man har lyst.

\[a + 1 - 5 \times 1 / 5 = a\]

\noindent
Skrives således: a + 1 - 5 \bs times 1 / 5 = a

\[ 2^2 = \sqrt{16} \]

\noindent
Skrives således: 2\^{}2 = \bs sqrt\{16\}

\[ 1^{22x} = \sqrt[38y]{1} \]

\noindent
Skrives således: 1\^{}\{22x\} = \bs sqrt[38y]\{1\}

\[ \frac{n!}{k!(n-k)!} = \binom{n}{k} \]

\noindent
Skrives således: \bs frac\{n!\}\{k!(n-k)!\} = \bs binom\{n\}\{k\}

\[ k_n = k_0 + k_{n + 1} \]

\noindent
Skrives således: k\_n = k\_0 + k\_\{n + 1\}

\[ \sum_{i=1}^{10} t_i \]

\noindent
Skrives således: \bs sum\_\{i=1\}\^{}\{10\} t\_i

\[ \int_{0}^{\infty} \mathrm{e}^{-x} \, \mathrm{d}x \]

\noindent
Skrives således: \bs int\_\{0\}\^{}\{\bs infty\} \bs mathrm\{e\}\^{}\{-x\} \bs , \bs mathrm\{d\}x\\

\noindent
Jeg håber, at du har fået det du skulle bruge. Ellers check de links jeg har skrevet i toppen af afsnittet.