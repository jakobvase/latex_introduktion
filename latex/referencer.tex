\chapter{Referencer}
Referencer i LaTeX virker på en anden måde end du måske er vant til. Det kan gøres på forskellige måder. Den første handler om når du refererer dig selv andre steder i dokumentet.

\section{Selvreferencer}
Når man gerne vil referere sig selv (som for eksempel i sætningen \myquote{Se f.eks. indledningen (side 3)}), bruger man i LaTeX to kommandoer. En til at markere et sted som noget der skal refereres til:\\

\bs label\{introduktion\}\\

\noindent
Og et til at referere til det:\\

\bs ref\{introduktion\}\\

\noindent
Ordet \myquote{introduktion}, som jeg skriver inden i \bs label og \bs ref, er det navn, som jeg husker stedet med. Så hvis jeg gerne vil referere til introduktionen, giver det mening at kalde den det. Hvis jeg ville referere til problemformuleringen i stedet, ville jeg kalde den det.

\noindent
Nu vil jeg give et eksempel:\\

\indent Her skriver jeg \bs label\{selvreferencer\}: \label{selvreferencer}\\
\indent Her skriver jeg \bs ref\{selvreferencer\}: \ref{selvreferencer}\\

\noindent
Ved \bs label står der ingenting, og ved \bs ref står sektionsnummeret der. Så nu kan jeg altså referere til denne sektion et hvilket som helst sted i dokumentet, ved at skrive således:\\

\indent Jeg skriver om referencer i sektion \bs ref\{selvreferencer\}\\
\indent Jeg skriver om referencer i sektion \ref{selvreferencer}\\

\noindent
Man kan også referere til sidetallet ved at skrive \bs pageref:\\

\indent Jeg skriver om selvreferencer på side \bs pageref\{selvreferencer\}\\
\indent Jeg skriver om selvreferencer på side \pageref{selvreferencer}\\

\noindent
Smart! Det eneste man skal være opmærksom på, er at give \bs label'sene forskellige navne, for ellers ved LaTeX ikke hvilken én man mener.

\section{Fodnoter, kommentarer og slutnoter}
Fodnoter indsættes ved at skrive:\\

\bs footnote\{Tekst som skal stå i fodnoten.\}\\

\noindent
Kommentarer indsættes ved at skrive:\\

\bs marginpar\{Tekst som skal stå i kommentaren.\}\\

\noindent
Her kommer eksempler på de to:\\

\indent Her skriver jeg en fodnote.\bs footnote\{Dette er en fodnote.\}\\
\indent Her skriver jeg en fodnote.\footnote{Dette er en fodnote.}\\
\indent Her skriver jeg en kommentar.\bs marginpar\{Dette er en kommentar.\}\\
\indent Her skriver jeg en kommentar.\marginpar{Dette er en kommentar.}\\

\noindent
Slutnoter er lidt mere omstændige, der skal man hente pakken \myquote{endnotes}, og derefter skrive denne kommando:\\

\bs let\bs footnote=\bs endnote\\

\noindent
Så altså, for at bruge slutnoter:\\

\indent \bs usepackage\{endnotes\}\\
\indent \bs let\bs footnote=\bs endnote\\

\noindent
Skal sættes ind i hovedfilen inden dokumentet begynder.