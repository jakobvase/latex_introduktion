\chapter{Bilagseksempel}

Dette er et bilag - både for eksemplets skyld og for at give lidt baggrundsinformation.

\section{At skrive specielle tegn}
Der findes en del tegn i LaTeX, som bliver brugt til at skrive kommandoer eller andet. Du er allerede stødt på "{\bs}" tegnet. Hvis man gerne vil skrive disse tegn i teksten skal man bruge specielle kommandoer, de kommer her:

\begin{table}[H]
\begin{tabular}{l | l || l | l}
\hline\noalign{\smallskip}
"\bs " & \bs textbackslash  & "\_" & \bs \_ \\
"\{" & \bs \{ & "\&" & \bs \& \\
"\}" & \bs \} & "\%" & \bs \% \\
"\$" & \bs \$ & "\#" & \bs \# \\
"\textgreater" & \bs textgreater & "\textless" & \bs textless \\
\noalign{\smallskip}
\hline
\end{tabular}
\end{table}

\noindent
Vær opmærksom på at \textbackslash, \textgreater og \textless skrives på en anden måde i "math mode", se kapitel \ref{chap:formler}.

Hvis du skriver komplicerede matematiske formler, vil du få brug for tegn jeg ikke har beskrevet her. En god side til at finde disse er

\noindent
\url{http://www.artofproblemsolving.com/Wiki/index.php/LaTeX:Symbols}

\noindent
og her er en pdf som har ALLE symbolerne:

\noindent
\url{http://www.tex.ac.uk/tex-archive/info/symbols/comprehensive/symbols-a4.pdf}.

\section{Fejl}
Nogle gange, når man skriver i LaTeX, kommer man forbi mærkelige fejl. Jeg beskriver de mest almindelige her, men ellers søg på nettet, så er der altid nogle som har prøvet det før.\\

\noindent
\textbf{Missing \$ inserted:} Dette betyder, at du har skrevet et symbol som kun findes i math-mode (se kapitel \ref{chap:formler}). For eksempel \bs backslash, \^ eller \_. For at bruge dem i text-mode, check listen over specielle tegn ovenfor.\\

\noindent
\textbf{Document already open:} Dette betyder, at pdf-filen som LaTeX gerne vil gemme, allerede er åben, og LaTeX kan derfor ikke få lov til at lave den. Check om du har pdf-filen åben i et andet program, og luk den der.\\

\noindent
\textbf{Underfull/Overfull \bs hbox:} Dette betyder, at LaTeX ikke er sikker på, om stedet er layoutet godt. I de fleste tilfælde er der ikke noget galt, men af og til ser noget mærkeligt ud. Det sker typisk når du bruger \bs url\{...\}, og navnet på hjemmesiden er mega langt, eller ved brug af billeder. For at rette det i \texttt{\bs url} kommandoen skal du ændre i dine \texttt{\bs usepackage} linier i toppen af dokumentet. Du skal skrive:\\

\indent \bs usepackage[hyphens]\{url\}\\
\indent \bs usepackage[hyphens]\{hyperref\}\\

\noindent
Du har sandsynligvis allerede tilføjet "hyperref" pakken. Det eneste du skal gøre er derfor at tilføje \texttt{\bs usepackage[hyphens]\{url\}} over "hyperref". Hvis du skriver "url" under "hyperref" får du en fejl. Hvis du stadig har problemer, se om du kan finde svar her: \url{http://tex.stackexchange.com/questions/3033/forcing-linebreaks-in-url}. Hvis problemet er billeder må du gå til google. Det er kompliceret.