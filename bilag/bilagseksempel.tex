\chapter{Bilagseksempel}

Dette er et bilag - både for eksemplets skyld og for at give lidt baggrundsinformation.

\section{At skrive specielle tegn}
Der findes en del tegn i LaTeX, som bliver brugt til at skrive kommandoer eller andet. Du er allerede stødt på "{\bs}" tegnet. Hvis man gerne vil skrive disse tegn i teksten skal man bruge specielle kommandoer, de kommer her:

\begin{table}[H]
\begin{tabular}{l | l || l | l}
\hline\noalign{\smallskip}
"\bs " & \bs textbackslash  & "\_" & \bs \_ \\
"\{" & \bs \{ & "\&" & \bs \& \\
"\}" & \bs \} & "\%" & \bs \% \\
"\$" & \bs \$ & "\#" & \bs \# \\
"\textgreater" & \bs textgreater & "\textless" & \bs textless \\
\noalign{\smallskip}
\hline
\end{tabular}
\end{table}

\noindent
Vær opmærksom på at \textbackslash, \textgreater og \textless skrives på en anden måde i "math mode", se kapitel \ref{chap:formler}.

Hvis du skriver komplicerede matematiske formler, vil du få brug for tegn jeg ikke har beskrevet her. En god side til at finde disse er

\noindent
\url{http://www.artofproblemsolving.com/Wiki/index.php/LaTeX:Symbols}

\noindent
og her er en pdf som har ALLE symbolerne:

\noindent
\url{http://www.tex.ac.uk/tex-archive/info/symbols/comprehensive/symbols-a4.pdf}.

\section{Fejl}
Fejl (pludselig math mode). (Errors også "pdf already open" og "underfull/overfull hbox").\todo{Skriv om fejl.}